\documentclass[a4paper,11pt]{article}

\usepackage{res}
\hypersetup{pdftitle={Dennies Bor - Postdoctoral CV}}

\begin{document}

\begin{tabular*}{\textwidth}{l@{\extracolsep{\fill}}r}
  \textbf{{\LARGE Dennies Bor}} & Email: \href{mailto:dbor@gmu.edu}{dbor@gmu.edu}\\
  GitHub: \href{https://github.com/denniesbor}{github.com/denniesbor} & LinkedIn: \href{https://linkedin.com/in/denniesbor}{linkedin.com/in/denniesbor} \\
  Website: \href{https://denniesbor.com}{denniesbor.com} & Scholar: \href{https://scholar.google.com/citations?user=mnet84cAAAAJ}{Google Scholar} \\
\end{tabular*}

\section{Professional Summary}
PhD candidate and computational scientist developing coupled physics--engineering--economic models for space-weather risk and infrastructure resilience. Focused on reproducible research software, uncertainty quantification, geospatial data products, and decision-relevant impact metrics.

\section{Skills Summary}
\resumeSubHeadingListStart
  \resumeSubItem{Coupled Systems Modeling}{Physics-based modeling, power-grid risk modeling, socio-economic impact modeling (Input-Output / CGE-style).}
  \resumeSubItem{Uncertainty Quantification}{Probabilistic simulation (Monte Carlo), scenario design, sensitivity analysis, validation workflows.}
  \resumeSubItem{Geospatial \& Remote Sensing}{Infrastructure mapping, GIS algorithms, satellite/overhead imagery processing, reproducible spatial pipelines.}
  \resumeSubItem{Scientific Computing}{Python scientific stack, numerical methods, optimization (Pyomo, IPOPT), scalable scenario studies.}
  \resumeSubItem{Research Software}{Open-source research artifacts, dashboards, reproducible workflows, version-controlled pipelines.}
\resumeSubHeadingListEnd

\section{Education}
\resumeSubHeadingListStart
  \resumeSubheading
    {George Mason University}{Fairfax, VA, USA}
    {PhD in Earth Systems and Geoinformation Sciences (Advisor: Dr. Edward Oughton)}{Sep 2023 -- Present}
    {\footnotesize\textit{\textbf{Relevant Coursework:} Computational Physics, Applied Electromagnetics, Atmospheric Physics, Earth Image Processing, GIS Algorithms, Spatial Computing}}\\
    {\footnotesize\textit{\textbf{GPA:} 3.96}}
  \resumeSubheading
    {Technical University of Kenya}{Nairobi, Kenya}
    {BEng in Aeronautical Engineering (First Class Honors)}{Sep 2013 -- May 2019}
\resumeSubHeadingListEnd

\section{Experience}
\resumeSubHeadingListStart
  \resumeSubheading{Graduate Research Assistant}{George Mason University, VA, USA}
  {Computational Modeling, Spatial Analysis, Infrastructure Resilience}{May 2022 -- Present}
  \resumeItemListStart
    \item Developed coupled hazard-to-impact models integrating numerical simulation, geospatial processing, and socio-economic impact estimation for infrastructure resilience.
    \item Built reproducible research software (data ingestion, scenario generation, simulation, postprocessing) and delivered results as open-source artifacts and dashboards.
    \item Conducted numerical simulations and uncertainty quantification using cloud computing workflows.
  \resumeItemListEnd

  \resumeSubheading{Engineering Intern}{Broglio Space Center, Malindi, Kenya}
  {Satellite Operations, Remote Sensing}{Aug 2018 -- Nov 2018}
  \resumeItemListStart
    \item Supported satellite tracking and telemetry for geospatial applications.
    \item Assisted in the maintenance and operation of RF communication systems.
  \resumeItemListEnd
\resumeSubHeadingListEnd

\section{Publications \& Preprints}
\resumeSubHeadingListStart
  \resumeSubItem{A Physics-Engineering-Economic Model Coupling Approach for Estimating Socio-economic Impacts of Space Weather (Primary Author)}
  {\href{https://arxiv.org/abs/2412.18032}{arXiv:2412.18032} \,|\, Code: \href{https://github.com/denniesbor/C-SWIM}{C-SWIM} \,|\, Dashboard: \href{https://denniesbor.me/portfolio/space-weather-grid}{space-weather-grid}}

  \resumeSubItem{A Reproducible Method for Mapping Electricity Transmission Infrastructure for Space Weather Risk Assessment (Co-Author)}
  {\href{https://arxiv.org/abs/2412.17685}{arXiv:2412.17685} \,|\, Dashboard: \href{https://denniesbor.github.io/spw-geophy-io/}{spw-geophy-io}}

  \resumeSubItem{GIC-Related Observations During the May 2024 Geomagnetic Storm in the United States (Co-Author)}
  {\href{https://arxiv.org/abs/2507.07009}{arXiv:2507.07009}}

  \resumeSubItem{Quantifying Political Polarization on Key Policy Issues Using Sentiment Analysis (Primary Author)}
  {\href{https://doi.org/10.48550/arXiv.2302.07775}{arXiv:2302.07775} \,|\, Results: \href{https://denniesbor.github.io/twitter_political_polarization/}{Dashboard} \,|\, Code: \href{https://github.com/denniesbor/twitter_political_polarization}{twitter\_political\_polarization}}
\resumeSubHeadingListEnd

\section{Projects \& Proposals}
\resumeSubHeadingListStart
  \resumeSubItem{Coupled Space Weather Impact Model (\textbf{C-SWIM})}
  {\href{https://github.com/denniesbor/C-SWIM}{GitHub} \,|\, \href{https://denniesbor.me/portfolio/space-weather-grid}{Results Dashboard}}

  \resumeSubItem{Reproducible Grid Mapping + Data Collection (\textbf{spw-geophy-io})}
  {\href{https://github.com/denniesbor/spw-geophy-io}{GitHub} \,|\, \href{https://denniesbor.github.io/spw-geophy-io/}{Dashboard}}

  \resumeSubItem{Space Radiation Risk for the Global Satellite Fleet (\textbf{sat-model})}
  {\href{https://github.com/denniesbor/sat-model}{GitHub}}

  \resumeSubItem{GIC Prediction Model Evaluation (\textbf{tfpy})}
  {\href{https://github.com/denniesbor/tfpy}{GitHub}}

  \resumeSubItem{Computer Vision for Power Substation Asset Detection (\textbf{substation-assets-identification})}
  {\href{https://github.com/denniesbor/substation-assets-identification}{GitHub}}
\resumeSubHeadingListEnd

\section{Professional Affiliations}
\resumeSubHeadingListStart
  \resumeSubheading
    {National Center for Atmospheric Research}{USA}
    {Early Career Faculty Innovators Program}{2023 -- 2025}
  \resumeSubheading
    {African Institute of Mathematical Sciences}{South Africa}
    {Africa Data Science Intensive Program}{2022}
\resumeSubHeadingListEnd

\section{References}
Available upon request.

\end{document}
