\documentclass[a4paper,11pt]{article}

\usepackage{res}
\hypersetup{pdftitle={Dennies Bor - Geospatial \& Remote Sensing CV}}

\begin{document}

\begin{tabular*}{\textwidth}{l@{\extracolsep{\fill}}r}
  \textbf{{\LARGE Dennies Bor}} & Email: \href{mailto:dbor@gmu.edu}{dbor@gmu.edu}\\
  GitHub: \href{https://github.com/denniesbor}{github.com/denniesbor} & LinkedIn: \href{https://linkedin.com/in/denniesbor}{linkedin.com/in/denniesbor} \\
  Website: \href{https://denniesbor.com}{denniesbor.com} & Scholar: \href{https://scholar.google.com/citations?user=mnet84cAAAAJ}{Google Scholar} \\
\end{tabular*}

\section{Professional Summary}
Geospatial and remote sensing computational scientist building reproducible spatial data products, infrastructure maps, and interactive web visualizations. Experienced in GIS algorithms, satellite/overhead imagery workflows, spatial statistics, and Python-based geospatial pipelines for decision-relevant analysis.

\section{Technical Skills}
\resumeSubHeadingListStart
  \resumeSubItem{Geospatial Computing}{GDAL, geopandas, rasterio; projections/CRS, vector--raster pipelines, spatial joins, network/topology processing.}
  \resumeSubItem{Spatial Analysis}{Hotspot/cold-spot analysis, clustering/LISA-style workflows, exposure/vulnerability mapping, quality control.}
  \resumeSubItem{Remote Sensing \& CV}{Satellite/overhead imagery processing; object detection for infrastructure assets (YOLO); feature extraction.}
  \resumeSubItem{Web Mapping}{Interactive dashboards; map-based UX for spatial QA and reporting (Leaflet/Cesium-style workflows).}
  \resumeSubItem{Scientific Python}{NumPy, SciPy, Pandas, Matplotlib; reproducible notebooks and batch pipelines.}
\resumeSubHeadingListEnd

\section{Education}
\resumeSubHeadingListStart
  \resumeSubheading
    {George Mason University}{Fairfax, VA, USA}
    {PhD in Earth Systems and Geoinformation Sciences (Advisor: Dr. Edward Oughton)}{Sep 2023 -- Present}

  \begin{itemize}[leftmargin=*,label={},itemsep=1pt,topsep=2pt]
    \item \small{\textbf{Selected Coursework:} Quantitative Methods; Remote Sensing; GIS Algorithms/Programming; Web-based GIS; Hyperspectral Imaging; Spatial Computing; Earth Image Processing; Geographic Information Systems.}
    \item \small{\textbf{GPA:} 3.85}
  \end{itemize}
  \vspace{-6pt}

  \resumeSubheading
    {Technical University of Kenya}{Nairobi, Kenya}
    {BEng in Aeronautical Engineering (First Class Honors)}{Sep 2013 -- May 2019}

\resumeSubHeadingListEnd

\section{Research Experience}
\resumeSubHeadingListStart
  \resumeSubheading{Graduate Research Assistant}{George Mason University, VA, USA}
  {Geospatial Modeling, Infrastructure Mapping, Spatial Analytics}{May 2022 -- Present}
  \resumeItemListStart
    \item Built geospatial datasets and algorithms to quantify infrastructure exposure and vulnerability under hazard scenarios.
    \item Developed reproducible GIS pipelines from open data sources; produced map-based products for analysis and stakeholder communication.
  \resumeItemListEnd

  \resumeSubheading{Engineering Intern}{Broglio Space Center, Malindi, Kenya}
  {Satellite Operations, RF Systems}{Aug 2018 -- Nov 2018}
  \resumeItemListStart
    \item Supported ground station operations and data acquisition workflows for Earth observation and geospatial applications.
  \resumeItemListEnd
\resumeSubHeadingListEnd

\section{Selected Projects}
\resumeSubHeadingListStart

  \resumeProjectHeading{\textbf{spw-geophy-io}: Grid Infrastructure Mapping + Dashboard \,|\, \href{https://github.com/denniesbor/spw-geophy-io}{spw-geophy-io}}{}
  \resumeItemListStart
    \item Open-source framework to map electricity transmission infrastructure for space-weather/GIC risk analysis with reproducible workflows.
    \item Interactive dashboard: \href{https://denniesbor.github.io/spw-geophy-io/}{spw-geophy-io dashboard}.
  \resumeItemListEnd

  \resumeProjectHeading{\textbf{substation-assets-identification}: Asset Detection in Overhead Imagery (YOLO) \,|\, \href{https://github.com/denniesbor/substation-assets-identification}{substation-assets-identification}}{}
  \resumeItemListStart
    \item Applied object detection to identify substation assets (e.g., transformers) from overhead imagery for infrastructure inventory workflows.
    \item Structured training/inference pipeline for repeatable experiments and evaluation.
  \resumeItemListEnd

  \resumeProjectHeading{\textbf{us\_broadband}: Spatial + Temporal ``Cold Spot'' Analysis \,|\, \href{https://github.com/denniesbor/us_broadband}{us\_broadband}}{}
  \resumeItemListStart
    \item Spatial analysis of broadband availability using threshold-based definitions and cluster diagnostics to identify persistent underserved regions.
    \item Produced maps and time-evolution summaries for spatial decision support.
  \resumeItemListEnd

  \resumeProjectHeading{\textbf{C-SWIM}: Space Weather Risk Dashboard (Geospatial Outputs) \,|\, \href{https://github.com/denniesbor/C-SWIM}{C-SWIM}}{}
  \resumeItemListStart
    \item Generated geospatial data products supporting hazard-to-impact analysis for space-weather-driven power-grid disruption.
    \item Results dashboard: \href{https://denniesbor.me/portfolio/space-weather-grid}{space-weather-grid}.
  \resumeItemListEnd

\resumeSubHeadingListEnd

\section{Selected Publications \& Preprints}
\resumeSubHeadingListStart
  \resumeSubItem{A Reproducible Method for Mapping Electricity Transmission Infrastructure for Space Weather Risk Assessment (Co-Author)}
  {\href{https://arxiv.org/abs/2412.17685}{arXiv:2412.17685} \,|\, Dashboard: \href{https://denniesbor.github.io/spw-geophy-io/}{spw-geophy-io}}

  \resumeSubItem{Socio-economic impact of electricity grid infrastructure failure due to severe space weather events (Primary Author)}
  {\href{https://arxiv.org/abs/2412.18032}{arXiv:2412.18032} \,|\, Code: \href{https://github.com/denniesbor/C-SWIM}{C-SWIM} \,|\, Dashboard: \href{https://denniesbor.me/portfolio/space-weather-grid}{space-weather-grid}}

  \resumeSubItem{GIC-Related Observations During the May 2024 Geomagnetic Storm in the United States (Co-Author)}
  {\href{https://arxiv.org/abs/2507.07009}{arXiv:2507.07009}}
\resumeSubHeadingListEnd

\section{Professional Development}
\resumeSubHeadingListStart
  \resumeSubheading
    {National Center for Atmospheric Research}{Boulder, CO, USA}
    {Early Career Faculty Innovators Program}{2023 -- 2025}
  \resumeSubheading
    {African Institute of Mathematical Sciences}{Cape Town, South Africa}
    {Africa Data Science Intensive Program}{2022}
\resumeSubHeadingListEnd

\section{References}
Available upon request.

\end{document}
